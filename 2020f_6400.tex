\documentclass[a4paper,landscape,11pt]{article}
%\documentclass[a4paper,11pt]{article} 
%\usepackage[T1,T2A]{fontenc}
\usepackage[utf8x]{inputenc}
%\usepackage[english,ukrainian,russian]{babel} 
\usepackage[english,russian]{babel} 
\usepackage{wrapfig}
\usepackage[table,xcdraw]{xcolor}
\usepackage{booktabs}
\usepackage{pifont}
\usepackage{graphicx}
\graphicspath{ {images/} }

\usepackage{tikz}
\usepackage{siunitx}
\usepackage[american,cuteinductors,smartlabels]{circuitikz}
\usetikzlibrary{decorations.text}

\usepackage{hyperref}

\usepackage{advdate}
%\usepackage{showframe} % для отладки позиции на странице
\usepackage{cancel}

%\usepackage{showframe}
%\setlength{\voffset}{-72pt} %отступ сверху - чтобы увидеть откомментарить \usepackage{showframe}
%\setlength{\voffset}{-36pt} %landscape
\setlength{\voffset}{-56pt} %landscape
\setlength{\footskip}{-120pt} %landscape
\setlength{\topmargin}{0pt} 
%\setlength{\headheight{1pt}
\setlength{\headsep}{0pt}
%\setlength{\hoffset}{-72pt} %landscape
\setlength{\hoffset}{-222pt} %landscape
\setlength{\marginparwidth}{0pt}
\setlength{\textwidth}{800pt} %landscape
\setlength{\textheight}{538pt} %landscape
\setlength{\footskip}{-60pt}



%\author{ Прокшин Артем \\
%\small ЛЭТИ\\
%\small \texttt{taybola@gmail.com}}

%\date{}
%abcdefghijklmnop


\newcommand*\OK{&\small \ding{51}$\!\!_\circ$} % начал защищать
\newcommand*\Ok{&\small \ding{51}$\!\!_\circ$} % начал защищатi
\newcommand*\ok{&{\small \ding{51}}} % присутствовал
\newcommand*\oK{&{\small \ding{51}?}} % присутствовал?
\newcommand*\no{&{\small }} % отсутствовал
\newcommand*\D{\tiny\ding{48}} % защита, defend
\newcommand*\da{&{\small\ding{48}$\!\!_1$}} % защита, defend
\newcommand*\dab{&{\small\ding{48}$\!\!^1_2$}} % защита, defend
\newcommand*\ab{&{\small\ding{48}$\!\!^1_2$}} % защита, defend
\newcommand*\ad{&{\small${}^1\!\!$\ding{48}$\!\!_4$}} % защита, defend
%\newcommand*\ab{&{\small\ding{48}$\!\!^1_2$}} % защита, defend
\newcommand*\bc{&{\small\ding{48}$\!\!^2_3$}} % защита, defend
\newcommand*\dabc{&{\small\ding{48}$\!\!^1_{23}$}} % защита, defend
\newcommand*\dabcd{&{\small\ding{48}$\!\!^{12}_{34}$}} % защита, defend
\newcommand*\ac{&{\small\ding{48}$\!\!^1_{23}$}} % защита, defend
\newcommand*\db{&{\small\ding{48}$\!\!_2$}} % защита, defend
\newcommand*\dc{&{\small\ding{48}$\!\!_3$}} % защита, defend
\newcommand*\dd{&{\small\ding{48}$\!\!_4$}} % защита, defend
\newcommand*\bd{&{\small${}^2\!\!$\ding{48}$\!\!^3_{4}$}} % защита, defend
\newcommand*\de{&{\small\ding{48}$\!\!_5$}} % защита, defend
\newcommand*\dE{&{\small${}^4\!\!\!$\ding{48}$\!\!_5$}} % защита, defend
\newcommand*\cd{&{\small\ding{48}$\!\!^3_4$}} % защита, defend
\newcommand*\dg{&{\small\ding{48}$\!\!_6$}} % защита, defend
\newcommand*\fg{&{\small${}^6\!\!$\ding{48}$\!\!_7$}} % защита, defend
\newcommand*\dH{&{\small\ding{48}$\!\!_8$}} % защита, defend
\newcommand*\gh{&{\small\ding{48}$\!\!^7_8$}} % защита, defend
\newcommand*\fh{&{\small\ding{48}$\!\!^7_{89}$}} % защита, defend 
\newcommand*\ce{&{\small${}^3\!\!$\ding{48}$\!\!_5$}} % защита, defend
\newcommand*\ef{&{\small${}^5\!\!$\ding{48}$\!\!_6$}} % защита, defend
%\newcommand*\dh{&{\small\ding{48}$\!\!_8$}} % защита, defend
\newcommand*\di{&{\small\ding{48}$\!\!_9$}} % защита, defend
\newcommand*\cdef{&{\small ${}^2_4\!\!$\ding{48}$\!\!^{3}_{5}$}} % защита, defend
\newcommand*\cde{&{\small ${}^2\!\!$\ding{48}$\!\!^{3}_{5}$}} % защита, defend
\newcommand*\efg{&{\small ${}^5\!\!$\ding{48}$\!\!^{6}_{7}$}} % защита, defend
\newcommand*\befgh{&{\small ${}_2^5\!\!$\ding{48}$\!\!^{6}_{78}$}} % защита, defend
\newcommand*\Dh{&{\small${}^4\!\!$\ding{48}$\!\!_8$}} % защита, defend
\newcommand*\cfg{&{\small ${}^3\!\!$\ding{48}$\!\!^{6}_{7}$}} % защита, defend
\newcommand*\fgh{&{\small ${}^6\!\!$\ding{48}$\!\!^{7}_{8}$}} % защита, defend
\newcommand*\bce{&{\small ${}^2\!\!$\ding{48}$\!\!^{3}_{5}$}} % защита, defend
\newcommand*\dO{&{\small\ding{48}$\!\!_{15}$}}
\newcommand*\Skip{\noindent\rule{0.3cm}{0.9pt}}


\begin{document}
%\thispagestyle{empty}
% or
\pagenumbering{gobble}
%\AdvanceDate[-1] % печатаю в субботу а нужна пятница
\vspace{-1cm}
\begin{center}\today\end{center}
\vspace{-0cm}
%\vspace*{1\baselineskip} %landscape
%\begin{table} \centering 
\hspace{2cm} % landscape
%\hspace{-2cm} % portrait
\newcommand*{\CS}{9pt} % ширина колонки
\begin{tabular}{p{7pt}|l|p{\CS}|p{\CS}|p{\CS}|p{\CS}|p{\CS}|p{\CS}|p{\CS}|p{\CS}|p{\CS}}
\multicolumn{11}{c}{Ведомость посещения занятий по преобразователям энергии студентами 6400 группы} \\
\toprule 
&&&&&&&&&&\\
&&&&&&&&&&\\
&&&&&&&&&&\\
&&&&&&&&&&\\
&&&&&&&&&&\\
&&&&&&&&&&\\
&&\rotatebox{90}{\rlap{\small 21 ноября}}
&\rotatebox{90}{\rlap{\small 28 ноября }}
&\rotatebox{90}{\rlap{\small 12 декабря }}
&\rotatebox{90}{\rlap{\small  }}
&\rotatebox{90}{\rlap{\small  }}
&\rotatebox{90}{\rlap{\small  }}
&\rotatebox{90}{\rlap{\small  }}
&\rotatebox{90}{\rlap{\small }}
&\rotatebox{90}{\rlap{\small }}
\\
% commands vi to copy/paste D :+19 ->>> p :-18 :w
\midrule
1\,&  Барановский Руслан          \no\ok\ok&&&&&\\
2\,&  Евстратов Владимир          \ok\no\ok&&&&&&\\
3\,&  Задорожнюк Даниил Булатович \no\ok\no&&&&&\\
4\,&  Иванченко Максим Вадимович  \no\ok\no&&&&&\\
5\,&  Маратов Мирас               \no\no\no&&&&&\\
\midrule
6\,&  Обама Омбеде Николас Серж   \ok\ok\ok&&&&&\\   % Obama Nicolas Serge
7\,&  Седельников Вячеслав        \ok\no\ok&&&&&\\
8\,&  Дегбун Уильям Михаэль       \ok\ok\ok&&&&&\\   % Michael William D
9\,&  Филатенков Павел Андреевич  \ok\ok\ok&&&&&\\ 
10\,& Фомина Елизавета            \ok\ok\ok&&&&&\\
\midrule
11\,& Халил Зейн                  \ok\ok\ok&&&&&\\   % Zain Khalil
12\,& Альмушреки Осама Абду Али   \ok\ok\no&&&&&\\   % Osamah Almushreqi
13\,& Саид Амир                   \ok\ok\no&&&&&\\
14\,& Уллах Вахаб                 \no\no\no&&&&&\\
15\,& Содан Крешимир              \no\no\no&&&&&\\
16\,& Масаве Паскаль Диоскори     \ok\no\ok&&&&&\\
\bottomrule
\end{tabular} 
\newpage
%
%\hspace{-3.1cm} %landscape
\begin{tabular}{l|llccc|cccc|ccccccc}
\multicolumn{10}{c}{выполнение лабораторнах работ, 6400 группа} \\
\toprule
&&T1&T1& T2&T2& T3&T3& T4&T4&T5&T5&T6&T6&L1&L2\\
\midrule
1\,&  Барановский Руслан         &21.12& 29.12& 21.12& 29.12& 21.12& 29.12& 22.12& 30.12& 21.12& 30.12\\
2\,&  Евстратов Владимир         &     &      &      &      &      &      &      &      &      &      &&&3.01&  3.01\\
3\,&  Задорожнюк Даниил          &19.12& 19.12& 19.12& 29.12& 25.12& 30.12& 23.12& 23.12& 25.12& 29.12&&&31.12\\ %Zadorozhnyuk D. B.
4\,&  Иванченко Максим Вадимович &12.10& 19.12& 05.11& 23.12& 15.11& 19.12& 15.11& 21.12& 22.12& 23.12\\ %Ivanchenko M.V.
5\,&  Маратов Мирас              &20.12& 29.12& 06.11& 29.12& 20.12& 30.12& 21.12& 30.12& 20.12& 30.12&&&30.12\\ %Maratov M.K.
\midrule
6\,&  Обама Омбеде Николас Серж  &10.10& 17.10& 21.11& 21.11& 05.12& 23.12& 22.12& 23.12& 22.12& 23.12&&&31.12&31.12\\   % OBAMA Nicolas Serge % very good messages and attendancy records (and appreciate no screeming and yelling)
7\,&  Седельников Вячеслав       &19.12& 19.12& 20.12& 29.12& 21.12&  --- & 21.12& 30.12& 24.12& 30.12& 29.12& 6.01&31.12&31.12 \\   % Sedelnikov V.I.
8\,&  Дегбун Уильям Михаэль      &10.10& 17.10& 21.11& 21.11& 21.11& 26.12& 22.12& 26.12& 22.12& 26.12&&&31.12&31.12\\   % Michael W.D
9\,&  Филатенков Павел           &30.09 &23.12& 30.09& 23.12& 30.09& 23.12& 30.09& 23.12& 30.09& 23.12\\
10\,& Фомина Елизавета           &12.12& 12.12& 20.12& 29.12& 20.12& 19.12& 19.12& 19.12& 20.12& 26.12&&& 26.12 \\  % Fomina Elizabeth A.
\midrule
11\,& Халил Зейн                 &12.12& 12.12&  --- &  --- & 26.12& 26.12& 25.12& 26.12& 25.12& 26.12&&& 26.12\\ % Zain khalil
12\,& Альмушреки Осама Абду Али  &10.10& 10.10& 17.10& 17.10& 21.11& 21.11& 21.12& 21.12& 21.12& 21.12\\   % Osamah Almushreqi
13\,& Саид Амир                  &     &      &      &      & 30.12& 30.12& 30.12& 30.12& 30.12& 31.12\\ % saeedamir
14\,& Уллах Вахаб                &&&&&&&&\\
15\,& Содан Крешимир             &&&&&&&&\\
16\,& Масаве Паскаль Диоскори    &20.12& 29.12& 26.12& 29.12& 20.12& 26.12& 26.12& 26.12& 30.12& 31.12    \\ % Pascal Masawe
\bottomrule
\end{tabular}

\subsection{Task 1}
\begin{itemize}
\item Ivanchenko -- miss archive info, $\beta$ is wrong
\item OBAMA Nicolas Serge -- angles $\alpha$, $\gamma$ are not marked, actualli initial version was screewed
\item Масаве Паскаль Диоскори -- miss to draw $\alpha,\gamma$
\item Маратов Мирас  -- miss to draw $\alpha,\gamma$ 
\end{itemize}

\subsection{Task 2 Covariant and contravariant}
\begin{itemize}
\item Ivanchenko -- $T_a$ covers whole interval, it's one possible solution, another -- is make $m_1$ symmetric
\item OBAMA Nicolas Serge -- very good, queckly fixed picture, archive info mistakely has author as Filatenkov P.A.
\item Маратов Мирас  --  miss archive info
\item Седельников -- in archive info autor is Alexandra (female)
\item Барановский  --  miss archive info
\item Масаве Паскаль Диоскори -- at figure exchange please m1 and m2
\end{itemize}

\subsection{Task 3 Boost converter}
\begin{itemize}
\item Ivanchenko -- catch formula 10.18 
" А так же отвечаю на вопрос. Мне кажется что формула для оценки емкости конденсатора (10.18) не верна, поскольку как написано в тексте выше энергия запасенная в катушке переходит в энергию поля конденсатора, однако это происходит не за период как указано в формуле, а лишь за то время , за которое катушка разряжается." 
$C_{out}> \frac{i_{out}}{f\cdot \Delta U_{out}}$-- this equation is an estimation.
As to me I estimate by decay capacitor by $R_{load}$ (with lianerization exponent).

\item OBAMA Nicolas Serge -- good, title pages was doubled, in archive Subject and Keywards were intendently missed, this is no good. 
\end{itemize}

\subsection{Task 4 Сhecking data for boost converter using graphs}
\begin{itemize}
\item Ivanchenko -- has archive info, Graphs are right and have his numbers.

\item OBAMA Nicolas Serge -- According to the student's list yours number is 6.(in the report picked up 12)

\item Zadorozhnyuk -- made an extra job: matlab Analytical Solution with consideration of initial conditions


\end{itemize}

\subsection{Task 5 Down converter}
\begin{itemize}
\item Ivanchenko -- fixed archive info, report does not have code for the model.
 amplitude of riples in $V_{out}$ is very small. ${\small \Delta}V_{out}$ should be roughly 10\% from $V_{out}$ as required. This lead to extra costs for capacitor.
                     Inductors and capacitors are expencive.
\item OBAMA Nicolas Serge -- this interval [.0000501 - .0000520]  instead of
\begin{verbatim}
Vs 4 0 PWL(0 0 .0000501 0 .0000520 1 .0002 1) r=0
\end{verbatim}
could be made shorter
\begin{verbatim}
Vs 4 0 PWL(0 0 .0000520 0 .00005201 1 .0002 1) r=0
\end{verbatim}

\item Sedelnikov --   amplitude of riples should be equal roughly 10\%

\item Zain Khalil -- archive info does not have author and subject, pictures have watermarks 'activate pirate software' , digits in reports have mixed locale: "1.2" and "1,2".
                     Program model is inconsistent with graphs. {\tt~C1~3~0~0.23F} -- 'u' sign was ommit here. $R_{Load}$ is not present at all.
\item Baranouski -- archive info is absent: author, subject, keywords. amplitude of riples in $V_{out}$ is very small.  ${\small \Delta}V_{out}$ should be roughly 10\% from $V_{out}$ as required. This lead to extra costs for capacitor.
\item Maratov --     amplitude of riples in $V_{out}$ is very small. ${\small \Delta}V_{out}$ should be roughly 10\% from $V_{out}$ as required. This lead to extra costs for capacitor.

\item Sedelnikov --  amplitude of both riples in $V_{out}$  and riples in $I_{L}$ are very small. amplitude of the riples should be roughly 10\% from correspondent average values.
                   Program model is absent in the report.
\item Zadorozhnyuk -- archive info in pdf file has ifrormation from 1-st task. Inspector should be different student not me.  amplitude of riples in $V_{out}$ is very small.
\item Maratov -- amplitude of both riples in $V_{out}$  and riples in $I_{L}$ are very small. amplitude of the riples should be roughly 10\% from correspondent average values.
                   Program model is absent in the report.
\item Saeed Amir -- used open software KiCad. This is good!
\end{itemize}

\subsection{Active Rectifier}
\begin{itemize}
\item Sedelnikov --  made model by Matlab, one fix - non-symmetric instead of $\alpha, \beta$ he used $A,B$. Beside of charhing battery also compensation of
reactive power could be done by the same active rectifier. 
\end{itemize}

\subsection{Lab \textnumero 1 Research of characteristics of power semiconductor devices}
\url{https://github.com/trot-t/2019-solar/blob/master/lab_en1.pdf}
\begin{itemize}
\item Sedelnicov -- at page5 you are not reach point $V_{zigzag}$ when current starts to inrcrease drammaticaly (the limit of the voltmeter V1 has 300V, or by the key below of the voltmeter V1 you could extend the limit to 600V)
The mesurements shoud be when capacitor $C_1$ is switch off (bad conditions $\equiv$ classification conditions). But your mesurements were finished at $40\times 4$V.
This dramatical increase of current should be mesured in VAC for both directions (direct and backward). And VAC should be taken at $V_{control}=0$.
Minimum value of $Vi_{zigzag}$ were current "folds" is used to figure out class of thyristor. 
Take into account since the voltage has a half-wave sinusoidal shape, conversion factor is $\pi$.
\end{itemize}


\subsection{Lab \textnumero 2 Research of unmanaged rectifiers and filters}
\url{https://github.com/trot-t/2019-solar/blob/master/lab_en2.pdf}

\subsection{Lab \textnumero 4 Research of 3-phase half-wave thyristor converters}
\url{https://github.com/trot-t/2019-solar/blob/master/lab_en4.pdf}
\begin{itemize}
\item OBAMA Nicolas Serge 
\begin{tikzpicture}
\draw[thin,->] (0,0) -- (8,0) node[below] {$I$};
\draw[thin,->] (0,0) -- (0,6) node[left] {$U$};
\draw[domain=0.7:6.7, thick, dashed, red, thick,postaction={decorate,decoration={text along path,text align=left,transform={yshift=2mm}, text={external characteristic}}}] plot(\x,{-0.2*\x+5}) node[right] {$\alpha=0$};
\draw[domain=0.7:6.7, thick, dashed, red] plot(\x,{-0.18*\x+3.33}) node[right] {$\alpha=30$};
\draw[domain=0.7:6.7, thick, dashed, red] plot(\x,{-0.16*\x+1.66}) node[right] {$\alpha=60$};
\draw[domain=0.7:6.7, thick, dashed, red] plot(\x,{-0.14*\x+0}) node[right] {$\alpha=90$};
%\draw[domain=0.7:7.2, thick,dashed, blue]  plot(\x,{0.6*\x+.5}) node[right] {how mesurements where taken};
\draw [thick,dashed, blue, thick,postaction={decorate,decoration={text along path,text align=center,transform={yshift=-4mm}, text={how mesurements where taken}}}] (0.7,{0.6*0.7+.5}) to ({7.2},{0.6*7.2+.5});

% points
\draw[fill,blue] ({ 1.166/0.76}, {0.6*1.166/0.76+.5} ) circle (3pt);
\draw[fill,blue] ({ 2.833/0.78}, {0.6*2.833/0.78+.5} ) circle (3pt);
\draw[fill,blue] ({ 4.5/0.8}, {0.6*4.5/0.8+.5} ) circle (3pt);

\end{tikzpicture}

\end{itemize}

\end{document}
