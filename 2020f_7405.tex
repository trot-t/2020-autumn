\documentclass[a4paper,landscape,11pt]{article}
%\documentclass[a4paper,11pt]{article} 
%\usepackage[T1,T2A]{fontenc}
\usepackage[utf8x]{inputenc}
%\usepackage[english,ukrainian,russian]{babel} 
\usepackage[english,russian]{babel} 
\usepackage{wrapfig}
\usepackage[table,xcdraw]{xcolor}
\usepackage{booktabs}
\usepackage{pifont}
\usepackage{graphicx}
\graphicspath{ {images/} }

\usepackage{tikz}
\usepackage{siunitx}
\usepackage[american,cuteinductors,smartlabels]{circuitikz}

\usepackage{hyperref}

\usepackage{advdate}
%\usepackage{showframe} % для отладки позиции на странице
\usepackage{cancel}

%\setlength{\voffset}{-72pt} %отступ сверху - чтобы увидеть откомментарить \usepackage{showframe}
\setlength{\voffset}{-36pt} %landscape
\setlength{\footskip}{-20pt} %landscape
\setlength{\topmargin}{0pt} 
%\setlength{\headheight{1pt}
\setlength{\headsep}{0pt}
\setlength{\hoffset}{-72pt} %landscape
\setlength{\marginparwidth}{0pt}
\setlength{\textwidth}{530pt} %landscape
\setlength{\textheight}{808pt} %landscape


%\author{ Прокшин Артем \\
%\small ЛЭТИ\\
%\small \texttt{taybola@gmail.com}}

%\date{}
%abcdefghijklmnop


\newcommand*\OK{&\small \ding{51}$\!\!_\circ$} % начал защищать
\newcommand*\Ok{&\small \ding{51}$\!\!_\circ$} % начал защищатi
\newcommand*\ok{&{\small \ding{51}}} % присутствовал
\newcommand*\oK{&{\small \ding{51}?}} % присутствовал?
\newcommand*\no{&{\small }} % отсутствовал
\newcommand*\D{\tiny\ding{48}} % защита, defend
\newcommand*\da{&{\small\ding{48}$\!\!_1$}} % защита, defend
\newcommand*\dab{&{\small\ding{48}$\!\!^1_2$}} % защита, defend
\newcommand*\ab{&{\small\ding{48}$\!\!^1_2$}} % защита, defend
\newcommand*\ad{&{\small${}^1\!\!$\ding{48}$\!\!_4$}} % защита, defend
%\newcommand*\ab{&{\small\ding{48}$\!\!^1_2$}} % защита, defend
\newcommand*\bc{&{\small\ding{48}$\!\!^2_3$}} % защита, defend
\newcommand*\dabc{&{\small\ding{48}$\!\!^1_{23}$}} % защита, defend
\newcommand*\dabcd{&{\small\ding{48}$\!\!^{12}_{34}$}} % защита, defend
\newcommand*\ac{&{\small\ding{48}$\!\!^1_{23}$}} % защита, defend
\newcommand*\db{&{\small\ding{48}$\!\!_2$}} % защита, defend
\newcommand*\dc{&{\small\ding{48}$\!\!_3$}} % защита, defend
\newcommand*\dd{&{\small\ding{48}$\!\!_4$}} % защита, defend
\newcommand*\bd{&{\small${}^2\!\!$\ding{48}$\!\!^3_{4}$}} % защита, defend
\newcommand*\de{&{\small\ding{48}$\!\!_5$}} % защита, defend
\newcommand*\dE{&{\small${}^4\!\!\!$\ding{48}$\!\!_5$}} % защита, defend
\newcommand*\cd{&{\small\ding{48}$\!\!^3_4$}} % защита, defend
\newcommand*\dg{&{\small\ding{48}$\!\!_6$}} % защита, defend
\newcommand*\fg{&{\small${}^6\!\!$\ding{48}$\!\!_7$}} % защита, defend
\newcommand*\dH{&{\small\ding{48}$\!\!_8$}} % защита, defend
\newcommand*\gh{&{\small\ding{48}$\!\!^7_8$}} % защита, defend
\newcommand*\fh{&{\small\ding{48}$\!\!^7_{89}$}} % защита, defend 
\newcommand*\ce{&{\small${}^3\!\!$\ding{48}$\!\!_5$}} % защита, defend
\newcommand*\ef{&{\small${}^5\!\!$\ding{48}$\!\!_6$}} % защита, defend
%\newcommand*\dh{&{\small\ding{48}$\!\!_8$}} % защита, defend
\newcommand*\di{&{\small\ding{48}$\!\!_9$}} % защита, defend
\newcommand*\cdef{&{\small ${}^2_4\!\!$\ding{48}$\!\!^{3}_{5}$}} % защита, defend
\newcommand*\cde{&{\small ${}^2\!\!$\ding{48}$\!\!^{3}_{5}$}} % защита, defend
\newcommand*\efg{&{\small ${}^5\!\!$\ding{48}$\!\!^{6}_{7}$}} % защита, defend
\newcommand*\befgh{&{\small ${}_2^5\!\!$\ding{48}$\!\!^{6}_{78}$}} % защита, defend
\newcommand*\Dh{&{\small${}^4\!\!$\ding{48}$\!\!_8$}} % защита, defend
\newcommand*\cfg{&{\small ${}^3\!\!$\ding{48}$\!\!^{6}_{7}$}} % защита, defend
\newcommand*\fgh{&{\small ${}^6\!\!$\ding{48}$\!\!^{7}_{8}$}} % защита, defend
\newcommand*\bce{&{\small ${}^2\!\!$\ding{48}$\!\!^{3}_{5}$}} % защита, defend
\newcommand*\dO{&{\small\ding{48}$\!\!_{15}$}}
\newcommand*\Skip{\noindent\rule{0.3cm}{0.9pt}}


\begin{document}
%\thispagestyle{empty}
% or
\pagenumbering{gobble}
%\AdvanceDate[-1] % печатаю в субботу а нужна пятница
\begin{center}\today\end{center}
\vspace*{1\baselineskip} %landscape

%\begin{table} \centering 
%\hspace{-6cm} % landscape
\hspace{-2cm} % portrait
\newcommand*{\CS}{9pt} % ширина колонки
\begin{tabular}{p{7pt}|l|p{\CS}|p{\CS}|p{\CS}|p{\CS}|p{\CS}|p{\CS}|p{\CS}|p{\CS}|p{\CS}}
%\multicolumn{16}{c}{График выполнения лабораторных работ студентами 8871 группы} \\ 
\multicolumn{11}{c}{Ведомость посещения занятий по преобразовательной технике тудентами 7405 группы} \\
\toprule 
&&&&&&&&&&\\
&&&&&&&&&&\\
&&&&&&&&&&\\
&&&&&&&&&&\\
&&&&&&&&&&\\
&&&&&&&&&&\\
&&\rotatebox{90}{\rlap{\small }}
&\rotatebox{90}{\rlap{\small 24 октября}}
&\rotatebox{90}{\rlap{\small 31 октября }}
&\rotatebox{90}{\rlap{\small 14 ноября }}
&\rotatebox{90}{\rlap{\small 26 декабря }}
&\rotatebox{90}{\rlap{\small }}
&\rotatebox{90}{\rlap{\small }}
&\rotatebox{90}{\rlap{\small }}
&\rotatebox{90}{\rlap{\small }}
\\
% commands vi to copy/paste D :+19 ->>> p :-18 :w
\midrule
1\,& Акулов Павел Александрович            &\no\ok\ok\ok&&&\\
2\,& Вахрамеев Василий Евгеньевич          &\ok\no\no\ok&&&\\
3\,& Геласимов Илья Геннадьевич            &\ok\ok\no\no&&&\\
4\,& Дмитриева Анастасия Евгеньевна        &\no\ok\ok\ok&&&\\
5\,& Дмитроченко Александр Александрович   &\ok\no\ok\ok&&&\\
\midrule
6\,& Жардемали Олжас                       &\ok\no\no\no&&&\\
7\,& Кондратьев Дмитрий Александрович      &\no\ok\no\no&&&\\
8\,& Коротин Владимир Вадимович            &\ok\no\no\ok&&&\\
9\,& Лютов Михаил Андреевич                &\no\ok\no\no&&&\\ 
10\,&Медведев Александр Александрович      &\ok\ok\no\no&&&\\
\midrule
11\,&Нефёдов Павел Олегович                &\ok\no\no\no&&&\\
12\,&Санников Алексей                      &\ok\no\no\ok&&&\\
13\,&Семенюк Максим Павлович               &\no\no\ok\no&&&\\
14\,&Бычков Роман Евгеньевич               &\no\no\o\ok&&&&\\
15\,&Морозов Александр Сергеевич           &\no\no\no\no&&&\\
\midrule
16\,&Эрдэнэцогт Хулан                      &\no\ok\ok\no&&&\\
17\,&Иванов Максим Олегович                &\no\no\no\ok&&&\\
18\,&Архангельский Юрий Дмитриевич         &\no\no\ok\no&&&\\ 
19\,&Ермолович Георгий Борисович           &\no\no\no\no&&&\\
20\,& Мурадов Руслан                       &\no\no\no\no&&&\\
\bottomrule
\end{tabular} 

\newpage
%
\hspace{-4.1cm} %landscape
\begin{tabular}{l|llccccccccccccc}
\multicolumn{10}{c}{выполнение лабораторнах работ, 7405 группа} \\
\toprule
&&Л1&Л1& Л2&Л2& Л3&Л3& Л4&Л4&Л5&Л5\\
\midrule
1\,& Акулов Павел Александрович            &26.12& 26.12&      &      & 30.12& 30.12& 30.12& 30.12& 30.12& 30.12\\
2\,& Вахрамеев Василий Евгеньевич          &29.09& 26.12& 30.12& 30.12& 30.12& 30.12&      &      & 30.12& 30.12\\
3\,& Геласимов Илья Геннадьевич            &&&&&&&&\\
4\,& Дмитриева Анастасия Евгеньевна        &29.09& 26.12&      &      & 28.12& 30.12& 28.12& 30.12& 28.12& 30.12\\
5\,& Дмитроченко Александр Александрович   &26.12& 26.12&      &      &&&&\\
\midrule
6\,& Жардемали Олжас   &&&&&&&&\\
7\,& Кондратьев Дмитрий Александрович      &&&&&&&&\\
8\,& Коротин Владимир Вадимович            &26.12& 26.12&      &      & 30.12& 30.12& 30.12& 30.12& 30.12& 30.12\\
9\,& Лютов Михаил Андреевич                &&&&&&&&\\
10\,&Медведев Александр Александрович      &26.12& 30.12&&&&&&\\
\midrule
11\,&Нефёдов Павел Олегович                &&&&&&&&\\
12\,&Санников Алексей                      &26.12& 26.12&      &      & 28.12& 30.12& 28.12& 30.12& 28.12& 30.12\\
13\,&Семенюк Максим Павлович               &&&&&&&&\\
14\,&Бычков Роман Евгеньевич               &29.09& 26.12& 27.12& 30.12& 27.12& 30.12& 27.12& 30.12& 27.12& 30.12\\
15\,&Морозов Александр Сергеевич           &&&&&&&&\\
\midrule
16\,&Эрдэнэцогт Хулан                      &&&&&&&&\\
17\,&Иванов Максим Олегович                &26.12& 30.12& 28.12& 30.12& 28.12& 30.12& 28.12& 28.12& 28.12& 30.12\\
18\,&Архангельский Юрий Дмитриевич         &&&&&&&&\\ 
19\,&Ермолович Георгий Борисович           &&&&&&&&\\
20\,& Мурадов Руслан                       &&&&&&&&\\
\bottomrule
\end{tabular}

\subsection{Лабораторная работа в лаборатории}
\subsection{Изменение заряда суперконденсатора (задание от 10 октября)}
\subsection{моделирование повышающего преобразователя прерывистый режим}
\subsection{задание от 17 октября до стр 5 (3 графика)}
\subsection{моделирование активного выпрямителя}


\end{document}
